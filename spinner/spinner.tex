% !TEX program = xelatex
% \documentclass[border=3mm]{standalone}
\documentclass{article}

\newif\ifxe\xetrue
\ifxe
\usepackage{fontspec}
\usepackage{realscripts}
\setmainfont[Numbers={Lining}]{Brill}
\newcommand\mysub[2]{\textit{#1}\textsubscript{#2}}
\else
\usepackage{mathpazo}
\newcommand\mysub[2]{$#1_{#2}$}
\fi

\usepackage[paperwidth=5.5in, paperheight=8in]{geometry}
\usepackage{tikz}
\usetikzlibrary{arrows,decorations.text,shapes}
\tikzset{
match/.style={thick,sloped,line cap=rect,shorten <=-1pt,shorten >=-1pt},
radial/.style={out=300,in=240},
transverse/.style={cross},
cross/.style={preaction={draw=white,-,line width=3pt,shorten <=20pt,shorten >=20pt}},
% Misc
>=latex',
brad/.style={draw,ultra thin,star,star points=9,inner sep=0.5mm},
gridline/.style={ultra thin,gray!15},
% Rounds
teamname/.style args={#1}{precise pin={#1}},
roundname/.style={left}, %draw,rounded rectangle,rounded rectangle east arc=none,
roundnum/.style={right}, %draw,rounded rectangle,rounded rectangle west arc=none,
% Sources for precise pin style:
% http://tex.stackexchange.com/a/49793/
% http://tex.stackexchange.com/a/43946/
precise pin/.style args={#1:#2}{
    % pin={[
    %     inner sep=0pt,
        label={[
            append after command={
                node [
                    inner sep=0pt,
                    at=(\tikzlastnode.#1),
                    anchor=#1+180
                ] {#2}
            }
        % ]center:{}}
        ]#1:{}}
    % ]#1:{}}
}
}
\def\myalph#1{\csname @Alph\endcsname{#1}}
\def\myroman#1{\csname @Roman\endcsname{#1}}

\begin{document}
\newcommand\nteams{10}
\newcommand\prefix{Team }
\newcommand\radius{3.3}
\newcommand\teamfmt[1]{\textbf{Team \myalph{#1}}}
\newcommand\sitefmt[1]{Site #1}
\newcommand\roundfmt[1]{\textit{#1}}


\begin{figure}
\centering
\begin{tikzpicture}%[scale=1.3]

\pgfmathsetmacro\nminusone{\nteams-1}
\pgfmathsetmacro\dtheta{360/(\nteams-1)}
\foreach \i in {1,...,\nminusone}{
	\pgfmathsetmacro\angle{\dtheta*\i} % 90+
	\def\effangle{\ifnum\i=0 -180\else\angle\fi}
	\def\effradius{\ifnum\i=0 0\else\radius\fi}
	\def\name{\teamfmt{\i}}
	\coordinate[teamname={\effangle:\name}] (\prefix\i) at (\angle:\effradius);
	\pgfmathsetmacro\theta{\dtheta/2+\angle}
	\node[roundnum,rotate=\theta] at (\theta:\effradius) {\roundfmt{\i}};
	\ifnum\i=1
	\node[roundname,rotate=\theta] at (\theta:\effradius) {\roundfmt{Round}};
	\fi
	% \draw (0:0) -- (\angle:3.5);
}
\coordinate (\prefix0);

\foreach \ia in {0,...,\nminusone}{
	\foreach \ib in {\ia,...,\nminusone}{
		\draw[gridline] (\prefix\ia) -- (\prefix\ib);
	}
}

% \node (\prefix0) {\prefix\nteams};
% Stick the brad in here
\node[brad] (0,0) {};
\node[yshift=-3em,xshift=-1em] (\prefix\nteams) at (0,0) {\teamfmt{\nteams}};
% \coordinate(\prefix0) at (\prefix0.north);
\draw[ultra thin] circle (\radius);

\def\arcdist{27}
\def\raisebase{0.125}
\path [decorate,decoration={text along path,text align=center,text={Rotate once after each round}}] (90+\arcdist:\radius+1+\raisebase) arc (90+\arcdist:90-\arcdist:\radius+1+\raisebase);
\draw[<-] (90+\arcdist:\radius+1) arc (90+\arcdist:90-\arcdist:\radius+1);

\clip circle (\radius);
\pgfmathtruncatemacro\novertwo{\nteams/2}
\foreach \i in {1,...,\novertwo}{
	% 0 1 -1 -1
	\pgfmathtruncatemacro\c{Mod(-1+( 1*\i),\nteams)}
	\pgfmathtruncatemacro\n{Mod( 0+(-1*\i),\nteams)}
	\ifnum\c=0\global\def\c{\nteams}\fi
	\def\bend{\ifnum\i=1 radial\else transverse\fi}
	\def\pos {\ifnum\i=1 0.5\else 0.5\fi}
	\def\startanchor {\ifnum\i=1 .300\else\fi}
	\draw[match,\bend] (\prefix\c\startanchor)
		to node[pos=\pos,above] {\sitefmt{\i}} (\prefix\n);
}
\end{tikzpicture}

\caption{A \nteams-team round-robin as rotations of a 1-factor of \mysub K{\nteams}}
\end{figure}
\vspace*\fill

\end{document}

% http://www.emba.uvm.edu/~jdinitz/preprints/design_tourney_talk.pdf
% http://www.sci.brooklyn.cuny.edu/~amotz/GC-ALGORITHMS/PRESENTATIONS/edge-coloring.pdf
